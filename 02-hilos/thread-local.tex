\section{Almacenamiento local al hilo}

\begin{frame}[t]{Variables locales a hilos}
\begin{itemize}
  \item Alternativa a \cppkey{static} como especificador de almacenamiento: \cppkey{thread\_local}.
    \begin{itemize}
      \item Una variable \cppkey{static} tiene una única copia compartida por todos los hilos.
      \item Una variable \cppkey{thread\_local} tiene una copia por cada hilo.
    \end{itemize}

  \mode<presentation>{\vfill\pause}
  \item \textmark{Tiempo de vida}: duración de almacenamiento de hilo (\emph{thread storage duration}).
    \begin{itemize}
      \item Se inicia antes de su primer uso en el hilo.
      \item Se destruye en la salida del hilo.
    \end{itemize}
\end{itemize}
\end{frame}

\begin{frame}[t]{Razones}
\begin{itemize}
  \item \textgood{Transformar} datos de almacenamiento estático a almacenamiento local al hilo.
    \begin{itemize}
      \item Variable \textmark{estática}: Una única instancia en la aplicación.
      \item Variable \textmark{local al hilo}: Una instancia por hilo.
      \item \textbad{Cuidado}: Se pasa a tener varias copias de la variable.
    \end{itemize}

  \mode<presentation>{\vfill\pause}
  \item Mantener \textmark{cachés de datos} \textgood{locales} a cada hilo (acceso exclusivo).
    \begin{itemize}
      \item \textemph{Cada hilo} mantiene su \textmark{propia tabla} de valores frecuentes.
      \item Importante en máquinas con caché de procesador separada y protocolos de coherencia.
    \end{itemize}
\end{itemize}
\end{frame}

\begin{frame}[t,fragile]{Contador de usos}
\mode<presentation>{\vspace{-1em}}
\begin{columns}[T]

\column{.5\textwidth}
\begin{lstlisting}
class C {
public:
  void f() {
    calcula();
    n++;
  }
  static int contador() { return n; }
private:
  static int n;
};

int C::n = 0;

void tarea() {
  C c;
  for (int i=0;i<max; ++i) {
    c.f();
  }
}
\end{lstlisting}

\pause

\column{.5\textwidth}
\begin{lstlisting}
void do {
  {
    std::jthread t1{f}; 
    std::jthread t2{f};
  }
  std::cout << "n = " << C::contador() << "\n";
}
\end{lstlisting}
\end{columns}
\end{frame}

\begin{frame}[t,fragile]{Contador de usos}
\mode<presentation>{\vspace{-1em}}
\begin{columns}[T]
\column{.5\textwidth}
\begin{lstlisting}
class C {
public:
  void f() {
    calcula();
    n++;
  }
  static int contador() { return n; }
private:
  thread_local static int n;
};

thread_local int C::n = 0;

void tarea() {
  C c;
  for (int i=0;i<max; ++i) {
    c.f();
  }
  return C::contador();
}
\end{lstlisting}

\pause

\column{.5\textwidth}
\begin{lstlisting}
void do {
  auto r = std::async(tarea);
  auto s = std::async(tarea);
  std::cout << "n = " << r.get() + s.get() << "\n";
}
\end{lstlisting}
\end{columns}
\end{frame}

\begin{frame}[t,fragile]{Caches de resultados}
\begin{lstlisting}
int calcula_clave(int x) {
  static std::map<int, int> cache;
  auto i = cache.find(x);
  if (i != cache.end()) return i->second;
  auto res = algoritmo_complejo(arg);
  cache[arg] = res;
  return res;
}

std::vector<int> genera_lista(std::vector<int> v) {
  std::vector<int> r;
  for (auto x : v) {
    r.push_back(calcula_clave(x));
  }
  return r;
}
\end{lstlisting}
\begin{itemize}
  \item Para cada x se ejecuta el algoritmo una única vez.
  \item Útil si valores repetidos y algoritmo lento.
\end{itemize}
\end{frame}

\begin{frame}[t,fragile]{Caches de resultados}
\begin{lstlisting}
int calcula_clave(int x) {
  thread_local std::map<int, int> cache;
  auto i = cache.find(x);
  if (i != cache.end()) return i->second;
  auto res = algoritmo_complejo(arg);
  cache[arg] = res;
  return res;
}

std::vector<int> genera_lista(std::vector<int> v) {
  std::vector<int> r;
  for (auto x : v) {
    r.push_back(calcula_clave(x));
  }
  return r;
}
\end{lstlisting}
\begin{itemize}
  \item Se evita sincronización.
  \item Puede que se repita algún cálculo.
\end{itemize}
\end{frame}
