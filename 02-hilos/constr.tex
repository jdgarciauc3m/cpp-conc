\section{Construcción e identidad}

\begin{frame}[t,fragile]{Construcción de hilos}
\begin{itemize}
  \item Un hilo se construye con una función y los argumentos que se debe pasar a la función.
    \begin{itemize}
      \item Plantilla con número variable de argumentos.
      \item Seguro en tipos.
    \end{itemize}
\begin{lstlisting}
void f();
void g(int, double);

std::thread t1{f}; // OK
std::thread t2{f, 1}; // Error: demasiados argumentos.

std::thread t3{g, 1, 0.5}; // OK
std::thread t4{g}; // Error: faltan argumentos.
std::thread t5{g, 1, "Hola"}; // Error: tipos incorrectos
\end{lstlisting}
\end{itemize}
\end{frame}

\begin{frame}[t,fragile]{Construcción y referencias}
\begin{itemize}
  \item El constructor de \cppid{thread} es una plantilla con argumentos variables.
\begin{lstlisting}
template <class F, class ...Args> 
explicit thread(F&& f, Args&&... args);
\end{lstlisting}
  \item El paso de parámetros a un hilo es por valor.
  \item Para forzar el paso de parámetros por referencia:
    \begin{itemize}
      \item Usar una función de ayuda para \cppid{reference\_wrapper}.
      \item Usar lambdas y capturas por referencia.
    \end{itemize}
\begin{lstlisting}
void f(registro & r);

void g(registro & s) {
  std::thread t1{f,s};
  std::thread t2{f, std::ref(s)};
  std::thread t3{[&] { f(s); }};
}
\end{lstlisting}
\end{itemize}
\end{frame}

\begin{frame}[t,fragile]{Construcción en dos etapas}
\begin{itemize}
  \item La construcción incluye el inicio de la ejecución del hilo.
    \begin{itemize}
      \item No hay operación separada para \emph{iniciar} la ejecución.
    \end{itemize}
\end{itemize}

\begin{columns}[T]

\column{.5\textwidth}
\begin{lstlisting}
struct productor {
  productor(cola<peticiones> & c);
  void operator()();
  // ...
};

struct consumidor {
  consumidor(cola<peticiones> & c);
  void operator()();
  // ...
};
\end{lstlisting}

\column{.5\textwidth}
\begin{lstlisting}
void f() {
  cola<peticiones> c;
  productor prod{c};
  consumidor cons{c};

  std::thread tp{prod};
  std::thread tc{cons};

  tp.join();
  tc.join();
}
\end{lstlisting}

\end{columns}
\end{frame}

\begin{frame}[t,fragile]{Hilo vacío}
\begin{itemize}
  \item El constructor por defecto crea un hilo sin tarea de ejecución asociada.
\begin{lstlisting}
thread() noexcept;
\end{lstlisting}
  \item Útil en combinación con el constructor de movimiento.
\begin{lstlisting}
thread(thread &&) noexcept;
\end{lstlisting}
  \item Se puede mover una tarea de ejecución de un hilo a otro.
    \begin{itemize}
      \item El hilo original se queda sin tarea de ejecución asociada.
    \end{itemize}
\begin{lstlisting}
thread crea_tarea();
thread t1 = crea_tarea();
thread t2 = move(t1);
\end{lstlisting}
\end{itemize}
\end{frame}

\begin{frame}[t]{Identidad de un hilo}
\begin{itemize}
  \item Cada hilo tiene un identificador único.
    \begin{itemize}
      \item Tipo \cppid{thread::id}.
      \item Si el \cppid{thread} no está asociado con un hilo \cppid{get\_id()} devuelve \cppid{id\{\}}.
      \item El identificador del hilo actual se obtiene con \cppid{this\_thread::get\_id()}.
    \end{itemize}

  \mode<presentation>{\vfill\pause}
  \item \cppid{t.get\_id()} devuelve \cppid{id\{\}} si:
    \begin{itemize}
      \item No se le ha asignado una tarea de ejecución.
      \item Ha finalizado.
      \item La tarea se ha movido a otro \cppid{thread}.
      \item Se ha desasociado (\cppid{detach()}).
    \end{itemize}
\end{itemize}
\end{frame}

\begin{frame}[t,fragile]{Operaciones sobre \texttt{thread::id}}
\begin{itemize}
  \item Es un tipo dependiente de la implementación, pero debe permitir:
    \begin{itemize}
      \item Copia.
      \item Operadores de comparación (\cppid{==}, \cppid{<}, ...).
      \item Salida mediante el operador \cppid{<<}.
      \item Transformación \emph{hash} mediante la especialización \cppid{hash<thread::id>}.
    \end{itemize}
\begin{lstlisting}
void imprime_id(thread & t) {
  if (t.get_id() == id{}) {
    cout << "Hilo no válido" << endl;
  }
  else {
    cout << "Hilo actual: " << this_thread::get_id() << endl;
    cout << "Hilo recibido: " << t.get_id() << endl;
  }
}
\end{lstlisting}
\end{itemize}
\end{frame}

