\section{Hilos}

\begin{frame}[t,fragile]{Lanzamiento de hilos}
\begin{itemize}
  \item Un hilo representado por la clase \cppid{std::thread}.
    \begin{itemize}
      \item Normalmente representa un hilo del SO.
    \end{itemize}
\begin{lstlisting}
void f1();
void f2();

void g() {
  std::thread t1{f1}; // Lanza un hilo que ejecuta f1()
  std::thread t2{f2}; // Lanza un hilo que ejecuta f2()

  t1.join(); // Espera a que t1 termine.
  t2.join(); // Espera a que t2 termine.
}
\end{lstlisting}
\end{itemize}
\end{frame}

\begin{frame}[t,fragile]{Objetos función}
\begin{itemize}
  \item Un \textgood{objeto función} es un tipo de datos que puede compartarse
        como una función.
\begin{lstlisting}[escapechar=@]
struct mi_funcion {
  mi_funcion(int x) : valor{x} {}
  void operator()() { procesa(valor_); }
  int valor_;
};

void test() {
  mi_funcion f{42};
  f(); // Ejecuta f.operator()()
  //...@\pause@
  std::thread t1(f); // Ejecuta f.operator()() en t1
  std::thread t2(mi_funcion{42}); // Sobre un temporal
  //...
  t1.join();
  t2.join();
}
\end{lstlisting}

\end{itemize}
\end{frame}

\begin{frame}[t,fragile]{Objetos compartidos}
\begin{itemize}
  \item Dos hilos pueden acceder a un objeto compartido.
  \item Posibilidad de carreras de datos.
\end{itemize}
\begin{lstlisting}
int x = 42;

void f() { ++x; }
void g() { x=0; }
void h() { cout << "Hola" << endl; }
void i() { cout << "Adios" << endl; }

void carrera() {
  std::thread t1{f}; std::thread t2{g};

  t1.join(); t2.join();

  std::thread t3{h}; std::thread t4{i};

  t3.join(); t4.join();
}
\end{lstlisting}
\end{frame}

\begin{frame}[t,fragile]{Paso de argumentos}
\begin{itemize}
  \item Paso de parámetros simplificado sin necesidad de \emph{casts}.
  \item Varias alternativas.
\end{itemize}

\begin{columns}[T]

\column{.3\textwidth}

\begin{lstlisting}
void f1(int x);

struct f2 {
  void operator()();
  f2(int px) : x{px} {}
  int x;
}

void f3(double x);
\end{lstlisting}

\column{.7\textwidth}
\begin{lstlisting}
void g() {
  std::thread t1{f1, 10}; // Ejecuta f1(10)
  std::thread t2{f2{20}}; // Ejecuta f2{20}.operator()()
  std::thread t3{[] { f3(1.5); } }; // Ejecuta una lambda

  t1.join();
  t2.join();
  t3.join();
}
\end{lstlisting}

\end{columns}

\end{frame}

\begin{frame}[t,fragile]{Comprobación de argumentos}
\begin{itemize}
  \item Se comprueba número y tipo de los argumentos.
\end{itemize}

\mode<presentation>{\vfill}
\begin{block}{Paso de argumentos}
\begin{lstlisting}
void f1(int x);
void f2(double x, double y);

void g() {
  std::thread t1{f1, 10}; // Ejecuta f1(10)
  std::thread t2{f1}; // Error
  std::thread t3{f2, 1.0} // Error
  std::thread t4{f2, 1.0, 1.0}; // Ejecuta f2(1.0,1.0)
  //...
  // Joins de hilos
\end{lstlisting}
\end{block}
\end{frame}
