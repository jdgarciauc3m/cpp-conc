\section{Tareas asíncronas}

\begin{frame}[t]{Tareas asíncronas y futuros}
\begin{itemize}
  \item Una tarea asíncrona permite el lanzamiento simple de la ejecución de una tarea:
    \begin{itemize}
      \item En otro hilo de ejecución.
      \item Como una tarea diferida.
    \end{itemize}

  \mode<presentation>{\vfill\pause}
  \item El lanzamiento de una tarea asíncrona devuelve un \cppid{std::future}.

  \mode<presentation>{\vfill\pause}
  \item Un futuro \cppid{f} tiene una operación \cppid{f.get()}
    \begin{itemize}
      \item Si un hilo devuelve un valor, la operación \cppid{get()} devuelve ese valor.
      \item Si un hilo lanza una excepción, la operación \cppid{get()} lanza esa excepción.
    \end{itemize}
\end{itemize}
\end{frame}

\begin{frame}[t,fragile]{Invocación de tareas asíncronas}
\begin{lstlisting}
#include <future>
#include <iostream>

double tarea(int i, int j);

int main() {
  std::future<double> r = std::async(tarea, 1, 10);
  std::future<double> s = std::async(tarea, 50, 100);
  otra_tarea();
  auto x = r.get();
  auto y = r.get();
  std::cout << "x = " << x << "\n";
  std::cout << "y = " << y << "\n";
  return 0;
}
\end{lstlisting}
\end{frame}
